\documentclass[12pt]{article}
\usepackage[spanish]{babel}
\usepackage[utf8]{inputenc}
\usepackage{amsmath, amssymb}
\usepackage{graphicx}
\usepackage{fancyhdr}
\usepackage{titlesec}
\usepackage{geometry}
\geometry{a4paper, margin=2.5cm}
\usepackage{float}

\setlength{\headheight}{14.5pt}
\addtolength{\topmargin}{-2.5pt}

\titleformat{\section}{\large\bfseries}{\thesection.}{1em}{}
\pagestyle{fancy}
\fancyhf{}
\rhead{INF245 - Laboratorio 3}
\lhead{Don Bit y Bitópolis}
\rfoot{\thepage}

\title{\textbf{INF245 – Laboratorio 3} \\ Don Bit y Bitópolis}
\author{
    Sebastián Richiardi Pérez – 202030555-2 – 201\\
    Gabriel Alejandro Toro Varela – 202204557-4 – 201
}
\date{02 de junio de 2025}

\begin{document}

\maketitle

\section{Introducción}
En este laboratorio se implementa un sistema digital en Logisim para simular los recorridos dentro de la ciudad digital Bitópolis. Se emplea lógica secuencial sin memorias ROM/RAM, únicamente con compuertas lógicas y flip-flops tipo D, mostrando paso a paso el trayecto en un display de 7 segmentos.

\section{Descripción del problema}
Don Bit construyó una ciudad digital donde los vehículos se guían por un código de 4 bits que define un nodo inicial en un grafo. Cada uno de los 16 códigos posibles activa una ruta única y secuencial hacia el nodo final común: F.

% -------------------------------------------------------
%    3. Diseño del sistema
% -------------------------------------------------------
\section{Diseño del sistema}

La FSM consta de:

1. Un registro de 4 bits que almacena el “estado actual” (el nodo en el grafo).  
2. Lógica combinacional de \textbf{próximo estado} (Next–State) que, para cada contenido actual del registro, produce los 4 bits del nodo siguiente.  
3. Un decodificador a 7 segmentos que, a partir de los 4 bits del estado, activa las líneas del display para mostrar la letra del nodo (A–P).  
4. Una señal de “Enable” que detiene la máquina en cuanto se alcanza el nodo final (F).

\subsection{Algoritmo general}

\begin{itemize}
  \item \textbf{Entrada inicial:} El usuario ingresa un código de 4 bits (b3\,b2\,b1\,b0), que corresponde a un número decimal entre 0 y 15. Cada valor 0–15 identifica un nodo inicial del grafo (\texttt{A}, \texttt{B}, …, \texttt{P}).
  \item \textbf{Estado inicial:} Al presionar “Reset”, el registro de estado carga directamente el código de 4 bits ingresado, es decir, el nodo de partida.
  \item \textbf{Transiciones:} En cada flanco de subida del reloj, si Enable = 1, el registro pasa de “Estado\_actual” a “Estado\_siguiente” de acuerdo con la ruta predefinida. Esa ruta se deriva del grafo de Bitópolis: cada estado (nodo) tiene exactamente un sucesor, salvo el nodo \texttt{F} (código 5), que es el “destino final” y se mantiene en sí mismo.
  \item \textbf{Detención:} Cuando el registro de estado contiene el código del nodo \texttt{F} (decimal\,5), la señal Enable pasa a 0 y la máquina deja de avanzar (queda “retenida” en \texttt{F}).
  \item \textbf{Visualización:} El decodificador 7 segmentos traduce los 4 bits del “Estado\_actual” a los segmentos a–g para que se muestre la letra correspondiente.  
\end{itemize}


\begin{center}
% \includegraphics[width=0.8\textwidth]{diagram.png}
\end{center}

Donde:
\begin{itemize}
  \item El bloque \textbf{Registro\_Estado (4\,D–FF)} guarda el nodo actual.  
  \item El bloque \textbf{Next\_State (Lógica combinacional 4 $\to$ 4)} calcula, a partir del contenido del registro, el código de los 4 bits del nodo siguiente.  
  \item El bloque \textbf{Decodificador\_7seg (4 $\to$ 7)} traduce el código binario del nodo (0–15) a los 7 bits que activan cada segmento del display.  
  \item La señal \textbf{Enable} se genera mediante comparación (Estado\_actual $==$ 5) y se emplea para inhabilitar el reloj al llegar a \texttt{F}.  
\end{itemize}

\newpage

\subsection*{Subcircuito \texttt{Decodificador 7 segmentos}}

% ---------------------------
% 3.4 Tabla de verdad del Decodificador 7 segmentos
% ---------------------------

\begin{table}[ht]
\centering
\begin{tabular}{|c|c|c||c|c|c|c|c|c|c|}
\hline
\textbf{Dec} & \textbf{Binario} & \textbf{Nodo} 
  & \textbf{a} & \textbf{b} & \textbf{c} & \textbf{d} & \textbf{e} & \textbf{f} & \textbf{g} \\
\hline
0  & 0000 & A & 1 & 1 & 1 & 0 & 1 & 1 & 1 \\  
1  & 0001 & B & 0 & 0 & 1 & 1 & 1 & 1 & 1 \\  
2  & 0010 & C & 1 & 0 & 0 & 1 & 1 & 1 & 0 \\  
3  & 0011 & D & 0 & 1 & 1 & 1 & 1 & 0 & 1 \\  
4  & 0100 & E & 1 & 0 & 0 & 1 & 1 & 1 & 1 \\  
5  & 0101 & F & 1 & 0 & 0 & 0 & 1 & 1 & 1 \\  
6  & 0110 & G & 1 & 0 & 1 & 1 & 1 & 1 & 0 \\  
7  & 0111 & H & 0 & 1 & 1 & 0 & 1 & 1 & 1 \\  
8  & 1000 & I & 0 & 1 & 1 & 0 & 0 & 0 & 0 \\  
9  & 1001 & J & 0 & 1 & 1 & 1 & 1 & 0 & 0 \\  
10 & 1010 & K & 1 & 0 & 1 & 0 & 1 & 1 & 1 \\  
11 & 1011 & L & 0 & 0 & 0 & 1 & 1 & 1 & 0 \\  
12 & 1100 & M & 1 & 0 & 1 & 0 & 1 & 0 & 1 \\  
13 & 1101 & N & 1 & 0 & 1 & 0 & 1 & 0 & 0 \\  
14 & 1110 & O & 1 & 1 & 1 & 1 & 1 & 1 & 0 \\  
15 & 1111 & P & 1 & 1 & 0 & 0 & 1 & 1 & 1 \\  
\hline
\end{tabular}
\caption{Tabla de verdad del decodificador de 7 segmentos para nodos A–P.}
\label{tab:7seg_decoder}
\end{table}

% ---------------------------
% 3.4.1 Mapas de Karnaugh para el Decodificador 7 segmentos
% ---------------------------
\bigskip

\noindent
\textbf{Mapa K‐map de segmento \textsf{a}}  
\[
\begin{array}{c|cccc}
\multicolumn{1}{c}{x_3x_2 \backslash x_1x_0} & 00 & 01 & 11 & 10 \\
\hline
00 & 1 & 0 & 0 & 1 \\
01 & 1 & 1 & 0 & 1 \\
11 & 1 & 1 & 1 & 1 \\
10 & 0 & 0 & 0 & 1 \\
\end{array}
\]
\vspace{1em}

\noindent
\textbf{Función minimizada para \(a\):}
\[
x_3x_2 + x_2\overline{x_1} + \overline{x_3}\,\overline{x_1}\,\overline{x_0} + x_1\,\overline{x_0}
\]

\noindent
\textbf{Mapa K‐map de segmento \textsf{b}}  
\[
\begin{array}{c|cccc}
\multicolumn{1}{c}{x_3x_2 \backslash x_1x_0} & 00 & 01 & 11 & 10 \\
\hline
00 & 1 & 0 & 1 & 0 \\
01 & 0 & 0 & 1 & 0 \\
11 & 1 & 0 & 1 & 1 \\
10 & 1 & 1 & 0 & 0 \\
\end{array}
\]
\vspace{1em}

\noindent
\textbf{Función minimizada para \(b\):}
\[
\overline{x_2}\,\overline{x_1}\,\overline{x_0} + x_3\,\overline{x_2}\,\overline{x_1} + \overline{x_3}\,x_1\,x_0 + x_2\,x_1\,x_0 + x_3\,x_2\,x_1
\]

\noindent
\textbf{Mapa K‐map de segmento \textsf{c}}  
\[
\begin{array}{c|cccc}
\multicolumn{1}{c}{x_3x_2 \backslash x_1x_0} & 00 & 01 & 11 & 10 \\
\hline
00 & 1 & 1 & 1 & 0 \\
01 & 0 & 0 & 1 & 1 \\
11 & 1 & 1 & 0 & 1 \\
10 & 1 & 1 & 0 & 1 \\
\end{array}
\]
\vspace{1em}

\noindent
\textbf{Función minimizada para \(c\):}
\[
\overline{x_2}\,\overline{x_1} + \overline{x_3}\,\overline{x_2}\,x_0 + \overline{x_3}\,x_2\,x_1 + x_3\,x_1\,\overline{x_0} + x_3\,x_2\,\overline{x_1}
\]

\noindent
\textbf{Mapa K‐map de segmento \textsf{d}}  
\[
\begin{array}{c|cccc}
\multicolumn{1}{c}{x_3x_2 \backslash x_1x_0} & 00 & 01 & 11 & 10 \\
\hline
00 & 0 & 1 & 1 & 1 \\
01 & 1 & 0 & 0 & 1 \\
11 & 0 & 0 & 0 & 1 \\
10 & 0 & 1 & 1 & 0 \\
\end{array}
\]
\vspace{1em}

\noindent
\textbf{Función minimizada para \(d\):}
\[
\overline{x_3}\,x_1\,\overline{x_0} + \overline{x_3}\,\overline{x_2}\,x_1 + x_2\,x_1\,\overline{x_0} + \overline{x_3}\,\overline{x_2}\,x_0 + x_3\,\overline{x_2}\,x_0 + \overline{x_3}\,x_2\,\overline{x_1}\,\overline{x_0}
\]

\noindent
\textbf{Mapa K‐map de segmento \textsf{e}}  
\[
\begin{array}{c|cccc}
\multicolumn{1}{c}{x_3x_2 \backslash x_1x_0} & 00 & 01 & 11 & 10 \\
\hline
00 & 1 & 1 & 1 & 1 \\
01 & 1 & 1 & 1 & 1 \\
11 & 1 & 1 & 1 & 1 \\
10 & 0 & 1 & 1 & 1 \\
\end{array}
\]
\vspace{1em}

\noindent
\textbf{Función minimizada para \(e\):}
\[
\overline{x_3} + x_2 + x_1 + x_0
\]

\noindent
\textbf{Mapa K‐map de segmento \textsf{f}}  
\[
\begin{array}{c|cccc}
\multicolumn{1}{c}{x_3x_2 \backslash x_1x_0} & 00 & 01 & 11 & 10 \\
\hline
00 & 1 & 1 & 0 & 1 \\
01 & 1 & 1 & 1 & 1 \\
11 & 0 & 0 & 1 & 1 \\
10 & 0 & 0 & 1 & 1 \\
\end{array}
\]
\vspace{1em}

\noindent
\textbf{Función minimizada para \(f\):}
\[
x_3\,x_1 + \overline{x_3}\,x_2 + \overline{x_3}\,\overline{x_2}\,\overline{x_1} + x_1\,\overline{x_0}
\]

\noindent
\textbf{Mapa K‐map de segmento \textsf{g}}  
\[
\begin{array}{c|cccc}
\multicolumn{1}{c}{x_3x_2 \backslash x_1x_0} & 00 & 01 & 11 & 10 \\
\hline
00 & 1 & 1 & 1 & 0 \\
01 & 1 & 1 & 1 & 0 \\
11 & 1 & 0 & 1 & 0 \\
10 & 0 & 0 & 0 & 1 \\
\end{array}
\]
\vspace{1em}

\noindent
\textbf{Función minimizada para \(g\):}
\[
\overline{x_3}\,\overline{x_1} + \overline{x_3}\,x_1\,x_0 + x_2\,\overline{x_1}\,\overline{x_0} + x_2\,x_1\,x_0 + x_3\,\overline{x_2}\,x_1\,\overline{x_0}
\]

\newpage

% ------------------------------------------------------------
% Subcircuito “NextState”: tabla de verdad, mapas de Karnaugh y funciones minimizadas
% ------------------------------------------------------------

\subsection*{Subcircuito \texttt{NextState}}

El subcircuito \texttt{NextState} recibe como entrada el código en binario de 4\,bits  
\((s_{3},s_{2},s_{1},s_{0})\) que representa el nodo actual (A…P) y produce como salida  
el código en binario de 4\,bits \((n_{3},n_{2},n_{1},n_{0})\) correspondiente al siguiente nodo en la ruta. 

% ------------------------------------------------------------
% 1) Tabla de verdad del subcircuito NextState
% ------------------------------------------------------------
\begin{table}[H]
  \centering
  \begin{tabular}{cccc|cccc}
    \multicolumn{4}{c|}{\bf Estado presente} & \multicolumn{4}{c}{\bf NextState}\\
    \multicolumn{4}{c|}{\((s_{3}\;\;s_{2}\;\;s_{1}\;\;s_{0})\)} & \multicolumn{4}{c}{\((n_{3}\;\;n_{2}\;\;n_{1}\;\;n_{0})\)}\\
    \hline
    0 & 0 & 0 & 0 & 0 & 0 & 0 & 1 \\ % A → B
    0 & 0 & 0 & 1 & 0 & 0 & 1 & 1 \\ % B → D
    0 & 0 & 1 & 0 & 0 & 0 & 0 & 1 \\ % C → B
    0 & 0 & 1 & 1 & 1 & 0 & 0 & 0 \\ % D → I
    0 & 1 & 0 & 0 & 0 & 0 & 1 & 1 \\ % E → D
    0 & 1 & 0 & 1 & 0 & 1 & 0 & 1 \\ % F → F
    0 & 1 & 1 & 0 & 0 & 1 & 1 & 1 \\ % G → H
    0 & 1 & 1 & 1 & 1 & 1 & 0 & 0 \\ % H → M
    1 & 0 & 0 & 0 & 1 & 0 & 0 & 1 \\ % I → J
    1 & 0 & 0 & 1 & 1 & 1 & 0 & 0 \\ % J → M
    1 & 0 & 1 & 0 & 1 & 1 & 0 & 1 \\ % K → N
    1 & 0 & 1 & 1 & 1 & 1 & 0 & 1 \\ % L → N
    1 & 1 & 0 & 0 & 1 & 1 & 1 & 0 \\ % M → O
    1 & 1 & 0 & 1 & 1 & 1 & 1 & 1 \\ % N → P
    1 & 1 & 1 & 0 & 0 & 1 & 0 & 1 \\ % O → F
    1 & 1 & 1 & 1 & 0 & 1 & 0 & 1 \\ % P → F
  \end{tabular}
  \caption{Tabla de verdad del subcircuito \texttt{NextState}. Cada fila indica la transición  
    desde el estado presente \((s_{3}\,s_{2}\,s_{1}\,s_{0})\) al siguiente \((n_{3}\,n_{2}\,n_{1}\,n_{0})\).}
\end{table}

\bigskip

% ------------------------------------------------------------
% 2) Mapas de Karnaugh
% ------------------------------------------------------------

\subsubsection*{Mapas de Karnaugh para cada bit de salida}

Se usarán mapas de Karnaugh 4×4 con la convención Gray en filas \((s_3\,s_2)\) y columnas \((s_1\,s_0)\).  

\paragraph*{(a) Mapa de Karnaugh para \(n_{3}(s_{3},s_{2},s_{1},s_{0})\).}

\begin{center}
\begin{tabular}{c|cccc}
  \multicolumn{1}{c}{} & \multicolumn{4}{c}{\((s_{1}\,s_{0})\)} \\[-2pt]
  \cline{2-5}
  \((s_{3}\,s_{2})\) & 00 & 01 & 11 & 10 \\
  \cline{1-5}
  00 & 0 & 0 & \bf1 & 0 \\  % minterm 3
  01 & 0 & 0 & \bf1 & 0 \\  % minterm 7
  11 & \bf1 & \bf1 & 0 & 0 \\% minterms 12, 13
  10 & \bf1 & \bf1 & \bf1 & \bf1 \\ % minterms 8,9,11,10
  \cline{1-5}
\end{tabular}
\end{center}

\noindent
\textbf{Agrupaciones:}
\begin{itemize}
  \item \(\{8,9,11,10\}\) (fila \(s_{3}s_{2}=10\) entera) \(\;\longrightarrow\; s_{3}\,\overline{s_{2}}\).
  \item \(\{8,9,12,13\}\) (\(\,(s_{3}s_{2})\in\{10,11\},\,(s_{1}s_{0})\in\{00,01\}\))  
    \(\;\longrightarrow\; s_{3}\,\overline{s_{1}}\).
  \item \(\{3,7\}\) (\((s_{3}s_{2})\in\{00,01\},\,(s_{1}s_{0})=11\))  
    \(\;\longrightarrow\; \overline{s_{3}}\,s_{1}\,s_{0}\).
\end{itemize}

\noindent
\textbf{Función minimizada para \(n_{3}\):}
\[
  n_{3} \;=\; 
    \underbrace{s_{3}\,\overline{s_{2}}}_{\text{(fila 10)}} 
    \;+\; 
    \underbrace{s_{3}\,\overline{s_{1}}}_{\substack{\text{(filas 10–11 en}\\(s_{1}s_{0})\in\{00,01\})}} 
    \;+\; 
    \underbrace{\overline{s_{3}}\,s_{1}\,s_{0}}_{\text{(columnas 11)}}.
\]

\bigskip

\paragraph*{(b) Mapa de Karnaugh para \(n_{2}(s_{3},s_{2},s_{1},s_{0})\).}

\begin{center}
\begin{tabular}{c|cccc}
  \multicolumn{1}{c}{} & \multicolumn{4}{c}{\((s_{1}\,s_{0})\)} \\[-2pt]
  \cline{2-5}
  \((s_{3}\,s_{2})\) & 00 & 01 & 11 & 10 \\
  \cline{1-5}
  00 & 0 & 0 & 0 & 0 \\ 
  01 & 0 & \bf1 & \bf1 & \bf1 \\ % minterms 5,7,6
  11 & \bf1 & \bf1 & \bf1 & \bf1 \\% minterms 12,13,15,14
  10 & 0 & \bf1 & \bf1 & \bf1 \\ % minterms 9,11,10
  \cline{1-5}
\end{tabular}
\end{center}

\noindent
\textbf{Agrupaciones:}
\begin{itemize}
  \item \(\{12,13,15,14\}\) (fila \(s_{3}s_{2}=11\) entera)  
    \(\;\longrightarrow\; s_{3}\,s_{2}.\)
  \item \(\{6,7,15,14\}\) (\((s_{3}s_{2})\in\{01,11\},\,(s_{1}s_{0})\in\{11,10\}\))  
    \(\;\longrightarrow\; s_{2}\,s_{1}.\)
  \item \(\{6,7,11,10\}\) (\((s_{3}s_{2})\in\{01,10\},\,(s_{1}s_{0})\in\{11,10\}\))  
    \(\;\longrightarrow\; s_{1}.\)
  \item \(\{9,13\}\) (\((s_{3}s_{2})\in\{10,11\},\,(s_{1}s_{0})=01\))  
    \(\;\longrightarrow\; s_{3}\,s_{0}.\)
  \item \(\{10,14\}\) (\((s_{3}s_{2})\in\{10,11\},\,(s_{1}s_{0})=10\))  
    \(\;\longrightarrow\; s_{2}\,s_{0}.\)
\end{itemize}

\noindent
\textbf{Función minimizada para \(n_{2}\):}
\[
  n_{2} \;=\; 
    \underbrace{s_{3}\,s_{2}}_{\{12,13,15,14\}} 
    \;+\; 
    \underbrace{s_{2}\,s_{1}}_{\{6,7,15,14\}} 
    \;+\; 
    \underbrace{s_{1}}_{\{6,7,11,10\}} 
    \;+\; 
    \underbrace{s_{3}\,s_{0}}_{\{9,13\}} 
    \;+\; 
    \underbrace{s_{2}\,s_{0}}_{\{10,14\}}.
\]

\bigskip

\paragraph*{(c) Mapa de Karnaugh para \(n_{1}(s_{3},s_{2},s_{1},s_{0})\).}

\begin{center}
\begin{tabular}{c|cccc}
  \multicolumn{1}{c}{} & \multicolumn{4}{c}{\((s_{1}\,s_{0})\)} \\[-2pt]
  \cline{2-5}
  \((s_{3}\,s_{2})\) & 00 & 01 & 11 & 10 \\
  \cline{1-5}
  00 & 0 & \bf1 & 0 & 0 \\   % minterm 1
  01 & \bf1 & 0 & 0 & \bf1 \\% minterms 4,6
  11 & \bf1 & \bf1 & 0 & 0 \\% minterms 12,13
  10 & 0 & 0 & 0 & 0 \\ 
  \cline{1-5}
\end{tabular}
\end{center}

\noindent
\textbf{Agrupaciones:}
\begin{itemize}
  \item \(\{4,12\}\) (fila \(s_{3}s_{2}=01,11\), columna “00”):  
    \[
      s_{1}=0,\;s_{0}=0,\;s_{2}=1,\;(s_{3}\text{ varía})  
      \;\longrightarrow\; s_{2}\,\overline{s_{1}}\,\overline{s_{0}}.
    \]
  \item \(\{12,13\}\) (fila \(s_{3}s_{2}=11\), columnas \(\{00,01\}\)):  
    \[
      s_{3}=1,\;s_{2}=1,\;s_{1}=0,\;(s_{0}\text{ varía})  
      \;\longrightarrow\; s_{3}\,s_{2}\,\overline{s_{1}}.
    \]
  \item \(\{6\}\) (aislado en fila “01”, columna “10”):  
    \[
      (s_{3},s_{2},s_{1},s_{0}) = (0,1,1,0)  
      \;\longrightarrow\; \overline{s_{3}}\,s_{2}\,s_{1}\,\overline{s_{0}}.
    \]
  \item \(\{1\}\) (aislado en fila “00”, columna “01”):  
    \[
      (s_{3},s_{2},s_{1},s_{0}) = (0,0,0,1)  
      \;\longrightarrow\; \overline{s_{3}}\,\overline{s_{2}}\,\overline{s_{1}}\,s_{0}.
    \]
\end{itemize}

\noindent
\textbf{Función minimizada para \(n_{1}\):}
\[
  n_{1} 
  \;=\; 
    \underbrace{\bigl(\overline{s_{3}}\,s_{2}\,\overline{s_{0}}\bigr)}_{\{4,6\}} 
    \;+\; 
    \underbrace{\bigl(s_{3}\,s_{2}\,\overline{s_{1}}\bigr)}_{\{12,13\}} 
    \;+\; 
    \underbrace{\bigl(\overline{s_{3}}\,\overline{s_{2}}\,\overline{s_{1}}\,s_{0}\bigr)}_{\{1\}}.
\]

\bigskip

\paragraph*{(d) Mapa de Karnaugh para \(n_{0}(s_{3},s_{2},s_{1},s_{0})\).}

\begin{center}
\begin{tabular}{c|cccc}
  \multicolumn{1}{c}{} & \multicolumn{4}{c}{\((s_{1}\,s_{0})\)} \\[-2pt]
  \cline{2-5}
  \((s_{3}\,s_{2})\) & 00 & 01 & 11 & 10 \\
  \cline{1-5}
  00 & \bf1 & \bf1 & 0 & \bf1 \\   % minterms 0,1,2
  01 & \bf1 & \bf1 & 0 & \bf1 \\   % minterms 4,5,6
  11 & 0    & \bf1 & \bf1 & \bf1 \\% minterms 13,14,15
  10 & \bf1 & 0    & \bf1 & \bf1 \\% minterms 8,10,11
  \cline{1-5}
\end{tabular}
\end{center}

\noindent
\textbf{Agrupaciones:}
\begin{itemize}
  \item \(\{0,1,4,5\}\) (filas “00–01”, columnas “00–01”):  
    \(\;s_{3}=0,\;s_{1}=0\;\longrightarrow\;\overline{s_{3}}\,\overline{s_{1}}\).
  \item \(\{0,2,4,6\}\) (imbuída: filas “00–01”, columnas “00–10” (envolvente)):  
    \(\;s_{3}=0,\;s_{0}=0\;\longrightarrow\;\overline{s_{3}}\,\overline{s_{0}}\).
  \item \(\{10,11,14,15\}\) (filas “10–11”, columnas “10–11”):  
    \(\;s_{3}=1,\;s_{1}=1\;\longrightarrow\;s_{3}\,s_{1}\).
  \item \(\{8,10\}\) (fila “10”, columnas “00–10”):  
    \(\;s_{3}=1,\;s_{0}=0\;\longrightarrow\;s_{3}\,\overline{s_{0}}\).
  \item \(\{13\}\) (aislado en fila “11”, columna “01”):  
    \(\;(s_{3},s_{2},s_{1},s_{0})=(1,1,0,1)\;\longrightarrow\;s_{3}\,s_{2}\,\overline{s_{1}}\,s_{0}.\)
\end{itemize}

\noindent
\textbf{Función minimizada para \(n_{0}\):}
\[
  n_{0} 
  \;=\; 
    \underbrace{\bigl(\overline{s_{3}}\,\overline{s_{1}}\bigr)}_{p_{1}} 
    \;+\; 
    \underbrace{\bigl(\overline{s_{3}}\,\overline{s_{0}}\bigr)}_{p_{2}} 
    \;+\; 
    \underbrace{\bigl(s_{3}\,s_{1}\bigr)}_{p_{3}} 
    \;+\; 
    \underbrace{\bigl(s_{3}\,\overline{s_{0}}\bigr)}_{p_{5}} 
    \;+\; 
    \underbrace{\bigl(s_{3}\,s_{2}\,\overline{s_{1}}\,s_{0}\bigr)}_{p_{6}}.
\]

\bigskip

% ------------------------------------------------------------
% 3) Resumen de las funciones lógicas
% ------------------------------------------------------------

\subsubsection*{Funciones lógicas minimizadas finales}

Finalmente, las cuatro salidas \((n_{3},n_{2},n_{1},n_{0})\) se describen mediante:

\[
\begin{aligned}
  n_{3}(s_{3},s_{2},s_{1},s_{0})
  &=\, s_{3}\,\overline{s_{2}}
    \;+\; s_{3}\,\overline{s_{1}}
    \;+\; \overline{s_{3}}\,s_{1}\,s_{0},\\[6pt]
  n_{2}(s_{3},s_{2},s_{1},s_{0})
  &=\, s_{3}\,s_{2}
    \;+\; s_{2}\,s_{1}
    \;+\; s_{1}
    \;+\; s_{3}\,s_{0}
    \;+\; s_{2}\,s_{0},\\[6pt]
  n_{1}(s_{3},s_{2},s_{1},s_{0})
  &=\, \overline{s_{3}}\,s_{2}\,\overline{s_{0}}
    \;+\; s_{3}\,s_{2}\,\overline{s_{1}}
    \;+\; \overline{s_{3}}\,\overline{s_{2}}\,\overline{s_{1}}\,s_{0},\\[6pt]
  n_{0}(s_{3},s_{2},s_{1},s_{0})
  &=\, \overline{s_{3}}\,\overline{s_{1}}
    \;+\; \overline{s_{3}}\,\overline{s_{0}}
    \;+\; s_{3}\,s_{1}
    \;+\; s_{3}\,\overline{s_{0}}
    \;+\; s_{3}\,s_{2}\,\overline{s_{1}}\,s_{0}.
\end{aligned}
\]


\section{Supuestos}
\begin{itemize}
    \item Las rutas están predefinidas y son únicas por código de entrada.
    \item No se requiere retroceso ni manejo de errores en tiempo real.
    \item El sistema parte desde reposo y no repite el ciclo una vez finalizado.
\end{itemize}

\section{Conclusión}

\end{document}

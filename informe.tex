\documentclass[12pt]{article}
\usepackage[spanish]{babel}
\usepackage[utf8]{inputenc}
\usepackage{amsmath, amssymb}
\usepackage{graphicx}
\usepackage{fancyhdr}
\usepackage{titlesec}
\usepackage{geometry}
\geometry{a4paper, margin=2.5cm}

\titleformat{\section}{\large\bfseries}{\thesection.}{1em}{}
\pagestyle{fancy}
\fancyhf{}
\rhead{INF245 - Laboratorio 3}
\lhead{Don Bit y Bitópolis}
\rfoot{\thepage}

\title{\textbf{INF245 – Laboratorio 3} \\ Don Bit y Bitópolis}
\author{
    Sebastián Richiardi Pérez – 20203055-2 – 201\\
    Gabriel Alejandro Toro Varela – 202204557-4 – 201
}
\date{02 de junio de 2025}

\begin{document}

\maketitle

\section{Introducción}
En este laboratorio se implementa un sistema digital en Logisim para simular los recorridos dentro de la ciudad digital Bitópolis. Se emplea lógica secuencial sin memorias ROM/RAM, únicamente con compuertas lógicas y flip-flops tipo D, mostrando paso a paso el trayecto en un display de 7 segmentos.

\section{Descripción del problema}
Don Bit construyó una ciudad digital donde los vehículos se guían por un código de 4 bits que define un nodo inicial en un grafo. Cada uno de los 16 códigos posibles activa una ruta única y secuencial hacia el nodo final común: F.

\section{Diseño del sistema}

\subsection{Algoritmo general}

\subsection{Tablas de verdad}

\subsection{Mapas de Karnaugh}

\subsection{Subcircuitos}

\begin{itemize}
    \item Nombre
    \item Función
    \item Entradas y salidas
    \item Interacción con otros módulos
\end{itemize}

\section{Simulación en Logisim}

\section{Pruebas}

\section{Supuestos}
\begin{itemize}
    \item Las rutas están predefinidas y son únicas por código de entrada.
    \item No se requiere retroceso ni manejo de errores en tiempo real.
    \item El sistema parte desde reposo y no repite el ciclo una vez finalizado.
\end{itemize}

\section{Conclusión}

\end{document}
